\chapter{La via Lattea}
La \emph{Via Lattea} \`e la galassia in cui noi abitiamo. Il sistema solare \`e posizionato in uno dei suoi bracci in una zona molto periferica. Classificandola in modo pi\`u tecnico, si tratta di una galassia \emph{spirale barrata}\footnote{Nome che indica come sono disposta le braccia della galassia}. Pu\`o essere suddivisa in alcuni zone principali che prendono il nome di
\begin{center}
\begin{tabular}{c c c c}
	\hline
	\hline
	Zona & Massa & R & Geometria \\
	\hline
	\hline
	Alone Stellare & $10^9M_{\odot}$ & $r\sim20kpc$ & Sferico\\
	Disco di Gas   & $10^{10}M_{\odot}$ & $r\sim25kpc$ & Disco $h=0.15kpc$\\
	Rigonfiamento Centrale & $10^{10}M_{\odot}$ & $(6\times2\times2)Kpc$ & Ellissoide\\
	Disco Stellare & $10^{11}M_{\odot}$ & $r\sim15kpc$ & Disco $h=1kpc$\\
	Alone Dark Matter & $10^{12}M_{\odot}$ & $r>60kpc?$ & Sferico\\
	\hline
\end{tabular}
\end{center}
Le stelle nella galassia sono organizzate in \emph{ammassi} che possono essere di due tipi: \emph{Ammassi Globulari} e \emph{Ammassi Aperti}. Gli ammassi globulari sono distribuiti nell'alone galattico, mentre gli ammassi aperti sono situati nel disco stellare. Le principali caratteristiche di entrambi sono riassumibili in
\begin{center}
\begin{tabular}{c c c c c c}
	\hline
	\hline
	Ammasso & $N^o$ Stelle & Dimensioni & Gas & Neb. Planetarie & $N^o$ Conosciuti \\
	\hline
	\hline
	Aperto & $10^3$ - $10^4$ & $10pc$ & S\`i & No & $10^3$\\
	Globulare & $10^4$ - $10^6$ & $(20$ - $100)pc$ & No & S\`i & 160\\
	\hline
\end{tabular}
\end{center}
\section{Dinamica degli Ammassi Globulari}
Gli ammassi globulari, essendo sistemi composti da molte stelle, viene spontaneo tentare di applicare i metodi della termodinamica per studiarne la dinamica. Non \`e possibile per\`o trattare l'ammasso globulare come un gas perfetto. La teoria di gas perfetto \`e una teoria non-interagente, mentre in questo caso si hanno interazioni stella-stella ed \`e di tipo a \emph{lungo-range}. Questo complica lo studio delle traiettorie delle singole particelle. Si adotta un approccio di tipo campo medio usando il \emph{teorema del viriale} che in questo caso si pu\`o applicare in quanto si ha a che
\begin{itemize}
	\item Particelle che si muovono in uno spazio limitato $V$.
	\item $G=\sum_{i=1}^Nr_ip_i$\`e una quantit\`a finita. 
	\item $G\to cost$.
\end{itemize}

Per la limitatezza di $G$ si ha che 
\newl{\abs{G} = \abs{\sum_{i=1}^N r_i p_i} \leq \left. \sum_{i=1}^N\abs{r_i} \cdot \abs{p_i} \right|_{V} \leq NR_vP}
dove $R_v$ \`e stato ipotizzato limitato, quindi $G$ \`e limitata. Pi\`u o meno nello stesso modo \`e possibile mostrare il comportamento di $G$ per $t\to+\infty$
\newl{
	\abs{\left\langle\frac{d}{dt}G \right\rangle} = \abs{\lim_{\tau\to+\infty}\frac{1}{\tau}\int_{0}^{\tau}\frac{d}{dt}G\, dt } = \abs{\lim_{\tau\to\infty}\frac{G(\tau)-G(0)}{\tau}} \leq \abs{\lim_{\tau\to+\infty} \frac{2NR_vP}{\tau} } = 0
}
quindi per $t\to+\infty$, $G$ tende ad una costante. A questo punto \`e possibile enunciare che \emph{Il viriale, dopo un certo tempo di rilassamento, \`e costante}. Partendo dall'enunciato si dedurranno tutte le propriet\`a principali del viriale.

\newtheorem*{mydef}{Enunciato del Viriale}
\begin{mydef}
	\newl{ \left\langle k_t \right\rangle = -\frac{1}{2}\left\langle \sum_{i=1}^N r_i\cdot F_i \right\rangle_t \text{ Dove } k_t = \sum_i^N k_i \text{ energia cinetica totale} \label{enun:vir}}
\end{mydef}

\begin{proof}[Dimostrazione]
Partendo dal risultato ottenuto precedentemente sulla derivata di $\left\langle G' \right\rangle=0$ \`e possibile scrivere
\newl{\left\langle \frac{d}{dt} \sum_{i=1}^Nr_i \cdot p_i \right\rangle = 0 }
facendo le derivate
\newl{\left\langle 2\sum_{i=1}^Nk_i + \sum_{i=1}^Nr_i\cdot F_i \right\rangle = 0 \nonumber\\
	\left\langle k_t \right\rangle = -\frac{1}{2}\left\langle \sum_{i=1}^N r_i\cdot F_i \right\rangle }
\end{proof}
Questo per un caso generale. Considerando il caso specifico di forze centrali si il \emph{Teorema del Viriale per Forze Centrali}.
\newtheorem*{mydef2}{Viriale per forze centrali}
\begin{mydef2}
Il teorema del viriale per un campo di forze centrali del tipo 
\newl{U(r)\sim \frac{1}{r^{\alpha}},}
si scrive nella forma
\newl{ \left\langle K \right\rangle = -\frac{\alpha}{2}\left\langle U \right\rangle_t .}
Nel caso specifico della forza di attrazione gravitazionala si ha che $\alpha=1$.
\end{mydef2}
\begin{proof}
Per dimostrarlo si parte dall'enunciato del viriale dato precedentemente in cui si sostituisce la forza di interazione gravitazione tra $N$ particelle. Sapendo quindi che
\newl{F_{i,j} = k \frac{r_i - r_j}{\abs{r_i-r_j} ^3 } , }
in particolar modo, la forza su una singola particella \`e pari a
\newl{F_i = \sum_{j=1,j\neq i}^N F_{i,j} = k \sum_{j=1,j\neq i} \frac{r_i - r_j}{\abs{r_i-r_j} ^3 } }
e sostituendo in Eq.~\ref{enun:vir} si ottiene
\newl{-\frac{1}{2} k \left\langle \sum_{i=1}^N\left( r_i  \sum_{j=1,j\neq i} \frac{r_i - r_j}{\abs{r_i-r_j} ^3 }\right)   \right\rangle = -\frac{1}{2} k \left\langle \sum_{i=1}^N \sum_{j=1,j\neq i} \frac{r_i\left(r_i - r_j\right)}{\abs{r_i-r_j} ^3 }   \right\rangle = \nonumber \\
	= -\frac{k}{2}\sum_{i=1}^N \sum_{j\neq i}^N \left( \frac{(r_i-r_j)^2}{\abs{r_i-r_j} ^3} +\cancel{ \frac{(r_i-r_j)(r_i+r_j)}{\abs{r_i-r_j} ^3}} \right)= -\frac{1}{2}\sum_{i=1}^N \sum_{j\neq i}^N \frac{k}{\abs{r_i-r_j} } = \nonumber\\ 
	= -\frac{1}{2} 2U = -\frac{1}{2}U.
}
\end{proof}
Usando il teorema del viriale con gli ammassi globulari \`e possibile fare stimare molte grandezze fisiche. Per avere degli ordini di grandezze che siano un minimo attendibili si consideri un ammasso globulare standard composto da $10^6$ stelle e di raggio $R=5pc$. Con questi valori \`e possibile calcolare
\begin{itemize}
	\item Energia potenziale dell'ammasso $U$;
	\item Componente media cinetica $K$;
	\item Velocita` quadratica media $\left\langle v^2 \right\rangle$;
	\item velocit\`a di fuga $v_{fuga}$;
	\item Tempo di rilassamento $T_rel$;
	\item Massa totale dell'ammasso $M$;
\end{itemize}
\newtheorem*{mydef3}{Calcolo energia potenziale}
\begin{mydef3}
\newl{U = -\frac{3}{5} \frac{GM^2}{R} = -2.5\cdot 10^{51} erg  }
\end{mydef3}

\newtheorem*{mydef4}{Calcolo componente cinetica media}
\begin{mydef4}
Sapendo $U$ posso calcolare la componente cinetica media usando il teorema del viriale
\newl{\left\langle K \right\rangle_t = -\frac{1}{2} \left\langle U \right\rangle = - 1.2\cdot 10^{51} erg}
\end{mydef4}
Da notare che l'energia totale del sistema \`e minore di zero, questo implica che il sistema \`e legato e stabile, quindi \`e ormai termalizzato.

\newtheorem*{mydef5}{Velocit\`a quadratica media}
\begin{mydef5}
	\newl{\frac{1}{2} M_{*} \cdot N \cdot \frac{1}{N}\sum_{i=1}^N v_i^2 = \frac{1}{2}M_{tot}v_{m}^2\nonumber \\
	v_m = \sqrt{\frac{3GM_{tot}}{5R}} = 16 km/s   }
\end{mydef5}

\newtheorem*{mydef6}{Velocit\`a di fuga}
\begin{mydef6}
	\newl{\frac{1}{2}M_{*} v_f^2 - G\frac{M_{*} M_{tot}}{R} = 0 \nonumber\\
		v_f=\sqrt{\frac{2GM_{tot}}{R}} = 29 km/s
	}
\end{mydef6}
Il fatto che $v_m < v_f$ \`e a riprova di quello ricavato nel calcolo dell'energia totale dell'ammasso, cio\`e che \`e un sistema legato e stabile. Sia il calcolo totale dell'energia del sistema che il calcolo della velocit\`a media a confronto con quella di fuga, suggeriscono che gli ammassi globulari siano sistemi di stelle stabili, legati e quindi, usando un termine termodinamico", termalizzati. \`E quindi interessante scoprire il tempo che un ammasso globulare impega per termalizzare. A questo scopo i prossimi calcoli saranno concentrati sul trovare un modello esaustivo di interazioni che spieghi il modo in cui questi ammassi stellari termalizzano. Il tempo che un ammasso stellare impiega per entrare in una condizione di stabilit\`a viene definito \emph{Tempo di Rilassamento}.

\newtheorem*{mydef7}{Determinazione del Tempo di Rilassamento}
\begin{mydef7}
	Dato che gli ammassi globulari sono realt\`a stabili, il tempo di rilassamento deve essere almeno inferiore all'et\`a dell'universo.
\end{mydef7}
Per determinare il tempo di rilassamento dell'ammasso \`e necessario dare una definizione appropriata di \emph{ammasso rilassato}. Una prima definizione che \`e possibile dare consiste nel dire che l'ammasso, formato da $N$ stelle, \`e rilassato dopo che ogni stella che lo costituisce ha fatto $N$ interazioni con le altre.

Da questa definizione segue la necessit\`a di definire quando c'\`e interazione tra due stelle. 
Il primo tentativo di descrivere l'interazione \`e attraverso un raggio di cut-off che chiameremo \emph{Raggio di Collisione}, cio\`e la distanza massima per far si che due stelle formino uno stato legato. In questo modo la definizione di raggio di collisione viene spontanea
\newl{\frac{1}{2}M_{*}v^2 = G\frac{M_{*}}{r_c}, \nonumber\\
	r_c = 2 \frac{GM_{*}}{v^2}.
}
Ora \`e necessario definire una probabilit\`a di interazione tra le stelle. Questa dipender\`a dalla velocit\`a della stella e dalla densit\`a dell'ammasso. Sia $n$ la densit\`a stellare dell'ammasso, questo vuol dire che in un volume finito di dimensione $V=\pi r^2 \Delta x$ ci sono $N_{*}=n(\pi r^2 \Delta x)$ stelle, quindi una stella che attraversa questo volume dell'ammasso globulare compie al pi\`u un numero di interazioni pari a $N_{int} = N_{*}$. Riscrivendo meglio il volume in funzione anche della velocit\`a si ottiene che
\newl{(\pi r^2 v \Delta t)n = N_{int} \implicaa \Delta t = \frac{N_{int}}{n \pi r^2 v}.}
Usando subito la definizione di raggio di collisione si ottiene
\newl{\Delta t = \frac{N_{int} v^3 }{4\pi G^2 M_{*}^2  n}, }
ma $N_{int} \sim 1$ iunterazione ad una stella. Usando la definizione della densit\`a stellare e la definizione di $G$ uscente dal viriale
\newl{n=\frac{N}{\frac{4}{3}\pi R^3}\,\,\,\,\,\,\, G = \frac{R v^2}{M_{*}N}}
e supponendo che $3/5 \sim 1/2$ si ottiene la scrittura finale per $\Delta t$
\newl{\Delta t \sim \frac{NR}{3v}.}
Inserendo i valori tipici di un ammasso globulare si ottiene un tempo rilassamento medio di
\newl{\Delta t \sim \frac{1}{3} \frac{10^6 \times 5pc}{16 km / s} \sim 100 Gyrs }
che \`e un risultato fisicamente sbagliato. Come osservato precedentemente, gli ammassi globulari sono sistemi legati, quindi termalizzati. Questo suggerisce che nella stima di $\Delta t$ sono state fatte approssimazioni troppo forti che hanno reso il modello inattendibile. L'approssimazione pi\`u forte che \`e stata fatta riguarda la definizione di \emph{interaione tra le stelle}. \`E stato considerato che una stella interagisce con un'altra stella solo al di sotto del suo \emph{raggio di collisione}. In questo modo \`e stato messo un cut-off all'interazione facendola diventare a corto range. Naturalmente l'interazione gravitazionale \`e tipo \emph{a lungo range} quindi il passo successivo \`e quello di estendere l'interazione considerandola a lungo range e ricalcolare il tempo di rilassamento.
\begin{figure}
		\centering
		\fbox{
		\begin{tikzpicture}[scale=2,auto=center]
				\draw [->] (0,0) -- (4,0);
				\node[] at (0.3,0.6) {$*$};
				\node[] at (1,1) {$*$};
				\node[] at (1.3,0.2) {$*$};
				\node[] at (1.6,-0.7) {$*$};
				\node[] at (1.9,-0.3) {$*$};
				\node[] at (2.2,0.5) {$*$};
				\node[] at (2.5,0.9) {$*$};
				\node[] at (2.9,0.3) {$*$};
				\node[] at (3.3,-0.3) {$*$};
				\node[] at (3.6,-0.5) {$*$};
				\node[] at (3.9,-1) {$*$};
				
				\node[] at (0.3,-0.6) {$*$};
				\node[] at (1,-1) {$*$};
				\node[] at (1.3,-0.2) {$*$};
				\node[] at (1.6,0.7) {$*$};
				\node[] at (1.9,0.3) {$*$};
				\node[] at (2.2,-0.5) {$*$};
				\node[] at (2.5,-0.9) {$*$};
				\node[] at (2.9,-0.3) {$*$};
				\node[] at (3.3,0.3) {$*$};
				\node[] at (3.6,0.5) {$*$};
				\node[] at (3.9,1) {$*$};

				\node[] at (5,0.1) {$v$ corto-range};

				\draw[color=red,->] plot [smooth] coordinates {(0,0) (1,0.3) (2,-0.4) (3,0.5) (4,0.3) };

				\node[color=red] at (5,0.5) {$v$ long-range};

				\draw[->] (1,0.3) -- (1,0);
				\draw[->] (2,-0.4) -- (2,0);
				\draw[->] (3,0.5) -- (3,0);
				\draw[->] (4,0.3) -- (4,0);

				\node[] at (3.25,0.2) {$\Delta v_{\perp}$};
	
		\end{tikzpicture}
		}
		\caption{Andamento della velocit\`a della stella di riferimento nell'ammasso globulare nei due diversi modelli: interazione a corto-range ed interazione a lungo-range}
\end{figure}

Il punto di questo approccio consiste nel valutare i vari $\Delta v_{\perp}$ dalla direzione iniziale considerando l'interazione a corto range. La distribuzione delle stelle \`e possibile considerarla random quindi $\left\langle\Delta v_{\perp}\right\rangle=0$, la grandezza che pu\`o dare informazioni interessanti \`e $\left\langle\Delta v_{\perp}^2\right\rangle$, perch\`e \`e legato all'energia che la stella acquista, aumenta nel tempo e diventa significativamente importante per $\left\langle\Delta v_{\perp}^2\right\rangle \sim v^2$. 
\begin{figure}
		\centering
		\fbox{
		\begin{tikzpicture}[scale=1.3,auto=center]
				\draw[->] (-0.6,0) -- (0.4,0); 
				\draw[dashed,->] (2,0) arc (270:360:2);
				\draw[->] (4,3) -- (4,4);
				\node[] at (-1.1,0) {MRU};
				\node[] at (4.7,3.8) {MRU};
				\node[] at (4.1,0.6) {MCU};
				\node[] at (0.5,0) {$*$};
				\node[] at (2,2) {$*$};
				\node[] at (4,4.1) {$*$};

		\end{tikzpicture}
		}
		\caption{Tipo di interazione per $r>r_c$}
\end{figure}

Caratterizzata l'interazione per $r>r_c$ come
\newl{
		M_{*} \frac{\Delta v_{\perp}}{\Delta t} = G\frac{M_{*}^2}{r^2}\nonumber \\
		\Delta v_{\perp} = G\frac{M_{*}}{rv}
}
da cui \`e possibile valutare $\left\langle\Delta v_{\perp}^2\right\rangle$ facendo una media pesata sui
\newl{
		\left\langle\Delta v_{\perp}^2\right\rangle = \int_{r_c}^R \left(2\pi r dr\right)\left(v\Delta t \right)n\left(\frac{GM_{*}}{rv}\right)^2 =  \frac{2\pi n G^2 M_{*}}{v} \Delta t \log\left(\frac{R}{r_c}\right) = \nonumber \\
		= \frac{2\pi n G^2 M_{*}}{v} \Delta t \log\left(\frac{R v^2}{G M_{*}}\right) \sim \frac{2\pi n G^2 M_{*}}{v} \Delta t \log\left(N\right) 
}
valuto ora il $\Delta t$ tale che il contributo di $\left\langle\Delta v_{\perp}^2\right\rangle\sim v^2$ quindi
\newl{
		\Delta t = \frac{v^3}{2\pi n G^2 M_{*}^2 \log(N)} \sim \frac{1}{12 \ln(N/2)} \frac{N R}{v} \sim 2 Gyrs < ETA_{universo}
}
quindi per un ammasso globulare tipico si ha un tempo tipico di rilassamento di circa $2Gyr$, risultato compatibile con le osservazioni precedenti. Per stimare l'et\`a di un ammasso globulare ci sono altri metodi quali turn-off point e attraverso la stima della \emph{Luminosity Function}.
\section{Ammassi Aperti}
Una seconda classe di ammassi presenti nella Via Lattea sono gli \emph{Ammassi Aperti}. Sono situati sul disco galattico e sono formati da stelle e solitamente anche da gas e polveri.

La popolazione stellare degli ammassi aperti arriva da statistiche fatta su circa $1000$ ammassi aperti censiti entro i $2kpc$ dalla terra. \`E difficile osservare ammassi aperti pi\`u distanti a causa delle polveri del disco galattico. Questo implica il fatto che il numero di ammassi aperti galattici che conosciamo oggi sono una forte sottostima, ma comunque permette di concludere ugualmente che la popolazione degli AA \`e molto pi\`u numeorsa di quella degli AG.
\begin{figure}
		\centering
		\includegraphics[scale=0.4]{turnM67.png}
		\caption{Diagramma HR di M67 e NGC188 messi a confronto. Il diverso punto in cui avviene il Turn-off indica l'et\`a diversa dell'ammasso}
		\label{ETA:AA}
\end{figure}

In Fig.~\ref{ETA:AA} \`e presentato il diagramma HR della popolazione di stelle di due diversi ammassi aperti. Valutando la posizione del \emph{turn-off} point \`e possibile stimare l'et\`a dell'ammasso. Con lo stesso ragionamento utilizzato in precedenza per gli ammassi globulari possiamo stimare il tempo di rilassamento anche degli ammassi aperti come
\newl{\Delta t_r \sim \frac{\sqrt{NR^3}}{12\log(N/2)\sqrt{GM_{*}}} \sim 10^9Gyr}
che indica in modo evidente che la gran parte degli ammassi aperti che si conoscono non \`e rilassata.
Gli ammassi aperti hanno et\`a molto giovane perch\`e essendo situati all'interno del disco stellare hanno una probabilit\`a abbastanza alta di interagire con oggetti particolarmente massivi che possono letteralmente distruggere l'ammasso e in generale non hanno massa sufficiente per evitare l'evaporazione. Grazie alla rotazione differenziale del piano galattico l'ammasso aperto pu\`o entrare in stretta interazione con oggetti particolarmente massivi e quindi venir distrutti dalle varie \emph{forze mareali} a cui sono sottoposti.
\section{Il Mezzo Interstellare}
L'evidenza fenomenologica dell'esistenza di un mezzo interstellare \`e data dalla presenza di ampie zone  opache sul piano galattico, con uno spessore che solitamente si aggira sui $100/500 pc$.
\begin{figure}
		\centering
		\fbox{
		\begin{tikzpicture}
				\draw[] plot [smooth] coordinates {(-5,0) (-1.2,0.1) (-0.6,0.222) (0,0.6)  (0.6,0.222) (1.2,0.1) (5,0) };
				\draw[] plot [smooth] coordinates {(-5,0) (-1.2,-0.1) (-0.6,-0.222) (0,-0.6)  (0.6,-0.222) (1.2,-0.1) (5,0) };
				\node[] at (4,0) {$\odot$}; 
				\draw[<->] (0,1) -- (4,1);
				\node[] at (2,1.2) {$\sim 8.5 kpc$};
				\node[] at (0,-1) {Centro Galattico};
				\node[] at (4,-1) {Sole};
				\draw[->] (4.5,0.3) -- (4.5,0.03);
				\draw[->] (4.5,-0.3) -- (4.5,-0.03);
				\node[] at (5.8,0.4) {Alone};
				\node[] at (5.8,0) {$\sim 500pc$};
		\end{tikzpicture}
		}
		\caption{Schematizzazione della \emph{Via Lattea} con riferimento alle posizioni dell'alone gassoso rispetto al Sole}
\end{figure}
L'evidenza principale della presenza del mezzo interstellare sono la presenza di \emph{Globuli o Nubi} che sono appunto caratterizzati da addensamenti di mezzo interstellare. Da terra, nella banda del visibile, percepiamo queste zone come aree scure nel cielo. In generale \`e definibile come mezzo interstellare tutto ci\`o che appartiene alla Via Lattea e sta tra le stelle.

I motivi per cui \`e importante studiare il mezzo interstellare (ISM) sono sopratutto due. Il primo riguarda il fatto che l'ISM compone gran parte della massa della galassia, il secondo motivo riguarda il fatto che nell'ISM si vengono a creare le condizioni ideale per la nascita di nuove stelle.
Questo implica il fatto che la Via Lattea \`e un'attiva Fucina di nuove stelle. Per questo motivo \`e quindi ovvio che la materia, all'interno della Via Lattea, subisce varie forme in funzione ai processi stellari o galattici che stiamo considerando. \`E utile a riguardo aver chiari i flussi barionici che contraddistingono una galassia come la nostra Via Lattea.
\begin{itemize}
		\item La Via Lattea prende dal mezzo intergalattico, materia per circa $0.5 M_{\odot}/yr$, e va ad accrescere il mezzo interstellare della galassia;
		\item Il mezzo interstellare ha una massa di circa $7\cdot 10^9 M_{\odot}$ ed \`e caratterizzato dai seguenti flussi barionici:
				\subitem LOSS: Formazione stellare $\sim 1.3 M_{\odot}/yr$;
				\subitem GAIN: Venti stellari / Neb.Planetarie/Supernovae $\sim 0.5 M_{\odot}/yr$
		\item Le stelle rappresentano $\sim 5\cdot 10^{10}M_{\odot}$
				\subitem $\sim 0.2 M_{\odot}/yr$ di stelle vengono trasformate in nane bianche / stelle di neutroni o buchi neri.
\end{itemize}
Si deduce che l'ISM \`e quasi in bilancio. Ha un guadagno di massa di circa $1.0 M_{\odot}/yr$ e una perdita di circa $1.3 M_{\odot}/yr$. C'\`e comunque una certa perdita che suggerisce che il mezzo interstellare, con l'andare del tempo, va a diminuire trasformandosi in stelle.
\section{Polveri Interstellari}
Il mezzo interstellare \`e caratterizzato da \emph{Gas}, \emph{Polveri}, \emph{Raggi Cosmici} e \emph{Radiazione ELM}. In queste note ci si occuper\`a soltanto dei primi due tipo di costituenti dell'ISM. Le polveri dell'ISM sono quelle che causano l'oscuramento di zone sul piano galattico (in banda V, nota come \emph{Estinzione}). Le polveri emettono principalmente nella zoma dell'IR. L'estinzione della radiazione elettromagnetica dipende principalmente dalle Polveri. Con il termine estinzione si intende
\begin{itemize}
		\item Diffusione: il granello di polvere, dalla forma irregolare, riflette i raggi in direzioni random ottenendo un effetto di diffusione;
		\item Assorbimento: Il granello di polvere pu\`o assorbire pare della radiazione elm incidente.
\end{itemize}

\begin{figure}
		\centering
		\fbox{
		\begin{tikzpicture}

				\draw[] plot [smooth] coordinates {(3.2,0.5) (3.3,0.9) (3.1,1.2) (2.8, 2) (2.3,3.4) (3.5,4)}; 		
				\draw[] (3.5,4) -- (4,2);
				\draw[] (4,2) -- (4,1);
				\draw[] (4,1) -- (3.2,0.5);
				\draw[->] (0.5,2.7) -- (2.5,2.7);
				\draw[->] (2.5,2.7) -- (0.5,0.5);
				\node[] at (0.5,2.9) {Diffusione};

				\draw[->] (6.5,2.5) -- (3.9,2.5);
				\node[] at (6.5,2.5) {Assorbimento};
		\end{tikzpicture}
		}
		\caption{Grano di polvere interstellare con i due effetti che caratterizzano l'}
\end{figure}
L'estinzione fa diminuire il flusso della radiazione elm in modo tale che
\newl{
		I(L)=e^{-\tau(L)} I_{0},
}
dove con $\tau(L)$ si indica il coefficienti di assorbiento che \`e uguale a
\newl{
		\boxed{
				\tau(L) = \sigma n L
		}
}
\subsection{Calcolo del coefficiente di estinzione $A$}
\begin{figure}
		\centering
		\fbox{
		\begin{tikzpicture}
				\draw[] plot [smooth] coordinates {(3.2,0.5) (3.3,0.9) (3.1,1.2) (2.8, 2) (2.3,3.4) (3.5,4)}; 		
				\draw[] (3.5,4) -- (4,2);
				\draw[] (4,2) -- (4,1);
				\draw[] (4,1) -- (3.2,0.5);
				\draw[color=green,->] (0.5,2) -- (2.8,2);
				\draw[color=green,->] (2.8,2) -- (1,0.5);
				
				\draw[color=red,->] (0.5,3) -- (2.35,3);
				\draw[color=red,->] (0.5,1) -- (3.25,1);

				\draw[->] (-3,2.5) -- (0,2.5);

				\draw[->] (5,2.5) -- (8,2.5);
				\node[] at (-1.5,2.7) {$I_{0}, m$};
				\node[] at (6.5,2.7) {$I(L),m'$};


				\foreach \i in {0,4,5} {
						\draw[->] (0.5,\i) -- (4.5,\i);
				}
		\end{tikzpicture}
		}
		\caption{Estinzione della radiazione ELM a causa delle polveri}
		\label{Esti}
\end{figure}
Come si nota in Fig.~\ref{Esti} viene presentato il modello di estinzione delal radiazione elm da parte delle polveri del mezzo interstellare. Senza polveri, rileverei da terra una certa magnitudine $m$. A causa dell'estinzione l'intensit\`a della radiazione si abbassa e si rileva a terra una magnitudine $m'>m$. L'obiettivo \`e quello di valutare in modo quantitativo l'estinzione attraverso il \emph{coefficiente di estienzione} $A$ definito come
\newl{A = m'-m = - 2.5 \log \frac{I(L)}{I_{0}} = 1.0857 \tau(L).}
In presenza di una certa estinzione, si ha che
\newl{m' = m+A = M + 5\log_{10}\frac{d}{10pc} + A}
quindi non \`e pi\`u sufficiente conoscere $M$ per ricavare $d$, va conosciuto anche $A$, quindi \`e necessario elaborare tecniche di misura di $A$.
\subsection{Tecniche di Misure di $A$}
Per misurare $A$ si pu\`o sfruttare il fatto che l'estinzione \`e in funzione a $\lambda$. Quello che si pu\`o fare sono diverse valutazioni della stessa porzione di cielo in varie lunghezze d'onda e notare come varia la trasparenza. Una delle tecniche pi\`u usate \`e quella della \emph{funzione di luminosit\`a}. Definisco $N'(m)$ il numero di stelle osservate, quindi se ne deduce una densit\`a
\newl{
		n=N'(m)dm
}
dove $n$ rappresenta appunto il numero di stelle comprese tra $m$ e $m+dm$. Plottando $\log N'(m)$ si ottiene il grafico in Fig.~\ref{LUM:FUN}.
\begin{figure}
		\centering
		\begin{tikzpicture}
				\draw[->] (0,0) -- (10,0);
				\draw[->] (0,0) -- (0,5);
				\draw[] (2,0) -- (4,5);
				\draw[] (6,0) -- (8,5);
				\node[] at (10,-0.3) {$m$};
				\node[rotate=90] at (-0.3,5) {$\log N'(m)$};

				\node[rotate=66] at (4.3,5) {senza estinzione};
				\node[rotate=66] at (8.3,5) {con estinzione};
		\end{tikzpicture}
		\caption{Andamento di $\log N'(m)$ in funzione di $m$ per due classi stellari molto simili in presenza o meno di estinzione}
		\label{LUM:FUN}
\end{figure}
Da notare che le due rette sono parallele in quanto sono state prese due calssi di stelle molto vicine quindi son state prese due porzioni di cielo con pi\`u o meno la stessa densit\`a stellare. Valutando lo shift sull'asse $m$ \`e possibile valutare $A$.

La cosa \`e applicabile anche osservando la stessa portzione di cielo a differenti lunghezze d'onda, un esempio \`e quello in fugura Fig.~\ref{PHOT:EXTIN}.
\begin{figure}
		\centering
		\includegraphics[scale=1.2]{Estinzione.jpg}
		\caption{Stessa porzione di cielo fotografata in varie bande. Si nota come l'estinzione sia molto pi\`u intensa per lunghezze d'onda piccole}
		\label{PHOT:EXTIN}
\end{figure}
si consideri $N_1$ il campione di stelle nella foto senza estinzione, $N_2$ il campione di stelle nella foto con estinzione. Per la definizione di funzione di luminosit\`a si ha che
\newl{
		N_1(m_0) = \int_{-\infty}^{m_0} N'(m')\, dm'\nonumber\\
		N_2(m_0 -A) = \int_{-\infty}^{m_0-A}N'(m')\,dm'.
}
Dato che \`e stata inquadrata la stessa porzione di cielo si ha quindi che $N_1 = N_2$. Se si riesce a valutare la funzione di liminosit\`a $N'(m)$ si ha direttamente una stima di $A$ come $N_2 - N_1 \sim A$.
Se $A > 6$ l'oscuramento delle stelle \`e troppo grande da non aver possibilit\`a di determinare la funzione di luminosit\`a. Se $A<1$ allora l'effetto di oscuramento \`e troppo piccolo per poter essere misurato.
\subsection{Estinzione in funzione di $\lambda$}
Come \`e possibile vedere in Fig.~\ref{PHOT:EXTIN} il fenomeno dell'estinzione \`e strettamente correlato alla lunghezza d'onda in cui lo si osserva. Per $\lambda$ piccole il fenomeno \`e molto accentuato, mentre nell'IR e nel lontano IR il fenomeno diventa molto meno intenso.

Per effetto dell'estinzione le stelle appaiono leggermente pi\`u rosse, fenomeno conosciuto come \textbf{\emph{Stellar Redding}} perch\`e passano pi\`u facilmente le $\lambda$ pi\`u lunghe. Questo porta un errore sistematico nella determinazione dell'indice di arrossamento della stella. La classe spettrale invece non subisce alterazioni in quato le righe di assorbimento non subiscono alcun tipo di alterazione. Supponendo di avere due osservazioni, la prima in banda V e la seconda in banda B \`e possibile scrivere
\newl{
		m_V=M_V + 5 \log\frac{d}{10pc} + A_V\nonumber\\
		m_B=M_B + 5 \log\frac{d}{10pc} + A_B,
}
il modulo di distanza dipende da
\newl{
		m_V-m_B = (M_V - M_B) + (A_V - A_B)
}
in cui $m_V-m_B$ \`e la quantit\`a che si misura direttamente, $M_V - M_B$ \`e possibile stimarla dalle righe spettrali (che non subiscono alterazioni a causa della polvere), sapendo le due quantit\`a appena descritte, \`e possibile facilmente ricavare la differenza dei coefficienti di estinzione $A_V-A_B$. Per ricavarne uno solo \`e necessario passare dalla valutazione della densit\`a stellare. In generale vale la regola
\newl{R_v \equiv \frac{A_V}{A_B-A_V}\sim3.1}
quindi
\newl{A_v \sim 3.1(A_B-A_V)}
\subsection{Caratteristiche misurate e misurabili di $A$}
Una volta determinato il coefficiente di estinzione $A$, \`e possibile dedurre propriet\`a \textbf{chimiche} e \textbf{geometriche} della nube. Riprendendo la definizione del coefficiente di estinzione
\newl{
		A=-2.5\log e^{\tau} \sim \tau = \sigma n L
}
\`e possibile capire quali siano le grandezze macroscopiche e microscopiche della polvere da cui dipende:
\begin{itemize}
		\item L = spessore della nube (grandezza macroscopica);
		\item n = densit\`a della nube (grandezza macroscopica);
		\item $\sigma$ = sezione d'urto totale, assorbimento + diffusione (grandezze microscopica).
\end{itemize}
Valutando direttamente $A$, se $\sigma$ \`e piccola si ha che $A \propto n L = N_{col}$ dove $N_{col}$ \`e la densit\`a colonnare, cio\`e il numero di particelle presenti in una colonna unitaria di lunghezza $L$. Da questo si deduce la densit\`a della nube ma non restituisce nessun tipo di informazione chimica/geometrica sui grani di polveri che la costutuiscono. Per avere delle informazioni sulla natura microscopica dei grani lo si pu\`o fare sfruttando il forte legame dell'estinzione alla lunghezza d'onda. \`E possibile definire una funzione $f(\lambda)$ che dipenda solo dalle quantit\`a legate alle propriet\`a microscopiche della nube
\newl{
		f(\lambda) = \frac{A(\lambda)}{A_v} = \frac{\cancel{nL}\sigma_{\lambda}}{\cancel{nL}\sigma_v}= \frac{\sigma(\lambda)}{\sigma_v.}
}

La $f(\lambda)$ ha delle particolari caratteristiche quali:
\begin{itemize}
		\item[i.]   Nel visibile e nell'IR si ha un comportamento $f(\lambda) \sim \frac{1}{\lambda}$
		\item[ii.]  La $f(\lambda)$ ha un picco nell'UV, compatibile con le osservazioni in cui si ha il massimo dell'estinzione.
		\item[iii.] Si vedono strutture nello spettro IR che corrispondono a picchi di $A$ relativi a righe di assorbimento negli spettri
\end{itemize}
Il tutto per\`o non trascurando il fatto che $f(\lambda)$ non caratterizza in modo preciso e univoco una nube, ma \`e fortemente legata alla \emph{direzione di osservazione}.
\begin{figure}
		\centering
		\fbox{
		\begin{tikzpicture}

				\draw[->] (0,0) -- (10,0);
				\draw[->] (0,0) -- (0,5);

				\draw[color=red] (9,4.5) -- (9.5,4.5);
				\node[] at (10.3,4.5) {Nube1};

				\draw[color=blue] (9,4) -- (9.5,4);
				\node[] at (10.3,4) {Nube2};

				\draw[color=green] (9,3.5) -- (9.5,3.5);
				\node[] at (10.3,3.5) {Nube3};


				\node[] at (10,-0.3) {$1/\lambda [\mu m ^{-1}]$};
				\node[rotate=90] at (-0.3,4) {$f(\lambda)=A(\lambda)/A_v$};

				\draw[color=blue] plot [smooth] coordinates {(0,0) (1,0.5) (2,0.9) (3,1.4) (6,2.7) (9,5) };
				\draw[color=red] plot [smooth] coordinates {(0,0) (1,0.5) (2,0.9) (3,1.4) (6,2.7) (9,4) };	
				\draw[color=green] plot [smooth] coordinates {(0,0) (1,0.5) (2,0.9) (3,1.4) (6,2.7) (9,3) };
				\draw[dashed] (3,-0.5) -- (3,5.5);
				\node[] at (1.5,4) {Visibile};
				\node[] at (4.5,4) {UV};

				\foreach \x in {0,1,2,3,4,5,6,7,8}{
					\node[] at (\x,-0.3) {$\x$};
				}
		\end{tikzpicture}
		}
		\caption{Comportamento della $f(\lambda)$ in funzione di $1/\lambda$. Per lunghezze d'onda piccole l'effetto dell'estinzione diventa importante permettendo di mettere in luce anche le differenze microscopiche delle varie nubi.}
\end{figure}

\subsection{Assorbimento e Diffusione}
L'albedo \`e quella parte dell'estinzione causata dalla diffusione. Considerando che l'estinzione \`e causata dai fenomeni dell'assorbimento e della diffusione \`e possibile scrivere
\newl{
		I(L) = I_{0} e^{-\tau}=I_{0} e^{-\tau\left(a_{\lambda diff} + a_{\lambda ass}\right)}
}
dove
\newl{
		a_{\lambda diff} + a_{\lambda ass} = 1.
}
Le \emph{Nebulose a Riflessione} sono tipici oggetti con $a_{\lambda diff}=0.6$ perch\`e sono caratterizzate da una nube di gas con nelle vicinanze una o pi\`u stelle molto luminose. La distanza delle stelle che la illuminano per\`o non \`e sufficiente ad ionizzare il gas, quindi si ha solo diffusione.\subsection{Dimensioni dei Grani dell'ISM}
Un ulteriore caratteristica fisica del mezzo interstellare che \`e possibile determinare \`e la dimensione media dei grani, $r_g$, di polvere che lo costituiscono. Per vari size dei grani si hanno differenti effetti, per esempio:
\begin{itemize}
		\item Se $r_g \gg \lambda$ allora si hanno gli effetti tipici dati dall'ottica geometrica. $A(\lambda) = cost.$ e quindi non si ha nessun tipo di fenomeno di diffusione.
		\item Se $r_g \ll \lambda$ allora la sezione d'urto tende a zero $\sigma \to 0$. $A(\lambda)$\`e piccolo e costante, non si ha interazione tra la radiazione e i grani.
		\item Se $r_g \sim \lambda$ allora il fenomeno della diffusione \`e importante e si ha una forte correlazione tra $A$ e $\lambda$.
\end{itemize}

Nella categoria dei \textbf{\emph{grani grossi}} rientrano tutti i grani con sezione del $\mu m - mm$. Questi generano uno spettro nell'IR e solitamente sono molecole di $SiO$, $SiO_2$ e ghiaccio.

Nella categoria dei \textbf{\emph{grani piccoli}} rientrano le polveri di sezione $1nm-10nm$. Sono caratterizzati da generare forte estinzione nell'UV e sono principalmente idrocarburi, carbonio elementare e carbonio sotto forma di grafite.
\subsection{Forma dei Grani dell'ISM}
La luce delle stelle non \`e polarizzata per\`o da terra si rileva che una buona percentuale di luce che arriva dalle stelle \`e polarizzata. Tanto pi\`u polarizzata quanto pi\`u la stella \`e lontana e tanto pi\`u ISM deve attraversare. Quindi misurare e caratterizzare la polarizzazione della luce delle stelle \`e un modo per caratterizzare il mezzo interstellare.

Si osserva in particolare che la luce proveniente dalle stelle ha polarizzazione caratterizzata dal piano $E$ parallelo al piano galattico. Questo implica che i grani assorbono principalmente la componente perpendicolare della polarizzazione. Questo fenomeno suggerisce una forma dei grani di tipo ellissoidale in cui si hanno elettroni liberi pi\`u in una direzione che in un'altra.

Si osserva anche il contributo delle polveri alla polarizzazione della luce delle stelle, \`e necessario ma non sufficente, infatti si ha che se $A_v$ \`e piccolo allora non si ha quasi polarizzazione, se $A_v$ \`e grande allora la polarizzazione delle stelle pu\`o essere sia molto bassa che molto grande.

Dato che i grani interagiscono con il campo elm, questo indica che non sono neutri. Il mezzo interstellare nella sua totalit\`a risulta invece essere neutro. Le modalit\`a con cui i grani catturano elettroni dall'ambiente circostante, sono principalmente due:
\begin{itemize}
		\item Elettroni liberi che si fissano alla superficie;
		\item Effetto fotoelettrico da fotoni UV (poco importante nel caso di $A_v$ grande).
\end{itemize}

Nel caso di elettroni liberi che si fissano sulla superficie, \`e possibile stimarne il numero che un grano pu\`o ospitarne.
\begin{figure}
		\centering
		\fbox{
		\begin{tikzpicture}
				\draw[] plot [smooth] coordinates {(3.2,0.5) (3.3,0.9) (3.1,1.2) (2.8, 2) (2.3,3.4) (3.5,4)}; 		
				\draw[] (3.5,4) -- (4,2);
				\draw[] (4,2) -- (4,1);
				\draw[] (4,1) -- (3.2,0.5);
				
				\node[] at (3.2,0.5) {$\odot$};
				\node[] at (3.1,1.2) {$\odot$};
				\node[] at (2.8, 2) {$\odot$};
				\node[] at (2.3,3.4) {$\odot$};
				\node[] at (3.5,4) {$\odot$};
				\node[] at (4,2) {$\odot$};
				\node[] at (4,1) {$\odot$};

				\draw[->] (0,1) -- (2,1.5);
				\node[] at (0,1) {$\odot$};
				\node[] at (0,1.5) {$E=\frac{3}{2}K_B T$};
		\end{tikzpicture}
		}
		\caption{Meccanismo di deposito degli eletttroni sulla superficie del grano}
		\label{GRANO:EL}
\end{figure}
Come \`e possibile vedere in Fig.~\ref{GRANO:EL} un elettrone "termico" raggiunge la superficie del grano con un'energia $E_k=3/2 K_B T$, se $E_k > U_{coulomb}$ si ha la cattura dell'elettrone sulla superficie. Per determinare il numero massimo di elettroni che si possono depositare su di un grano basta quindi uguagliare $U_C$ e $E_k$,
\newl{ \frac{N e^2}{r_g} = \frac{3}{2}K_B T \implicaa N = \frac{3}{2} K_B r_g \frac{T}{e^2}\sim 1.  }
In realt\`a questo valre di $N$ \`e sottostimato. Considerando le code ad alta velocit\`a degli elettroni il numero di elettroni che si possono depositare su di un grano diventa di $N\sim10$.

\subsection{Temperatura dei grani}
Supponendo che i grani siano scaldati dalle stelle vicine possiamo considerare la stima della temperatura come se tutta l'energia arrivi da una stella di una certa $L$ e raggio $R$ posta ad una certa distanza $d$ dal grano. Ricordando che la luminosit\`a di una stella vista come corpo nero sferico \`e:
\newl{L=4 \pi R^2 \sigma T^4}
e la frazione di potenza che colpisce il grano \`e data da
\newl{
		f=\frac{\pi r_g^2}{4\pi d^2} = \frac{1}{4} \left(\frac{r_g}{d}\right)^2
}
Se il grano ha albedo $a$, assorbe una potenza pari a 
\newl{
		P_{abs}=f L (1-a) = (1-a) r_g^2 \sigma T^4 \left(\frac{\pi R^2}{d^2}\right)
}
dove $(\pi R^2)/d^2 = \Omega_{*}$ \`e l'angolo solido della stella visto dal grano.
La potenza rilasciata dal grano, se fosse un corpo nero perfetto sarebbe
\newl{
		P_{rad} = 4\pi r_g^2 \sigma T_g^4
}
ma avendo albedo $a$, la potenza irradiata sar\`a sempre scalata da
\newl{
		P_{rad}=(1-a)4\pi r_g^2\sigma T_g^4.
}
Se la polvere ha raggiungo la temperatura $T_g$ di equilibrio, si ha che $P_{rad} = P_{abs}$ e quindi
\newl{
		(1-a)4\pi r_g^2\sigma T_g^4 = (1-a) r_g^2 \sigma T^4 \left(\frac{\pi R^2}{d^2}\right)\nonumber\\
		T_g=T\sqrt{\frac{R}{2d}}
}
quindi, la temperatura dei grani, non dipende ne dalla loro forma ne dal loro albedo.

\subsection{Grani in Rotazione}
Molte evidenze suggeriscono che i grani siano in rotazione. Se il grano \`e costituito da materiale paramagnetico e ruota con momento di dipolo disallineato rispetto alla direzione del campo magnetico galattico $B_{gal}$, si ha una forza di torsione di richiamo che ovviamente tende ad allineare $\mu_{B}$ con $B_{gal}$. Se l'asse di rotazione del grano \`e gi\`a parallelo a $B_{gal}$ allora non si ha alcuna forza di richiamo. In generale i grani sono in rotazione e con tutti il loro $\mu_{B}$ allineato a $B_{gal}$, fenomeno conosciuto col nome di \emph{Rilassamento Paramagnetico}.

\textbf{Radiazione anomala a 32GHz ... Non si capisce bene il contributo alla radiazione totale, ma centrano le PAH che danno contributo appunto a 32GHz.}

Per avere un modello realistico che descriva il comportamento del mezzo interstellare bisogna considerare altri effetti per esempio:
\begin{itemize}
		\item Collisioni con ioni;
		\item Accoppiamento tra $\mu$ e il campo elm;
		\item Assorbimento ed emissione di fotoni;
		\item Formazioni di molecoled di $H_2$;
		\item Vari tipi di emissione di fotoni (rinculo, modi roto-vibrazionali, ...);
		\item ecc..
\end{itemize}
Modelli pi\`u dettagliati prevedono che
\newl{
		\boxed{
			\langle\omega^2\rangle \leq 3 K_B \frac{T}{I},
	}
}
che da come risultato sempre un'emissione sulle decine di GHz.

Oltre a questo, c'\`e il modello di Hensley che cerca una relazione tra 
\newl{
		\frac{F_{\nu}^{sd}}{M_{dust}}\propto q_{pah}
}
anche se le misure indicano invece che le grandezze $F_{\nu}^{sd}$ e $q_{pah}$ sono anticorrelate.

\section{Gas Interstellare (ISG)}
Le osservazioni che conducono all'idea della presenza di un gas interstellare sono principalmente tre: La prima riguarda il fatto che i sistemi binari non mostrano effetto doppler. La seconda riguarda le linee spettrali delel stelle lontane che sono molto piu` marcate di quelle vicine e sono molto pi\`u strette (compatibili con una $T\sim100K$). Questo ha suggerito la presenza di gas interstellare.

Non si osserva presenza di idrogeno interstellare nel \emph{visibile}. A causa della bassa temperatura non ci sono righe di Balmer, quindi il fatto che non sia visibile non dice nulla sula fatto che ci sia o meno. Gli elementi che sono stati osservati e quindi sono sicuramente presenti sono Calcio e Sodio ma sono state osservate anche molecole $CH$, $CN$, $CH^+$. Queste ultime implicano una bassa densit\`a del gas e una bassa temperatura perch\`e molecole cariche come $CH^+$, in condizioni di laboratorio si neutralizzano subito, mentre $CH$ e $CN$ sono fortemente reattive. 

\`E comunque lecito aspettarsi che la componente di idrogeno atomico sia predominante nel gas interstellare, anche se questo non \`e osservabile nel visibile. Le righe Lymann sono escluse dall'osservazione perch\`e vengono tutte assorbite dalle polveri del mezzo interstellare, le osservazioni che ci permettono di osservare l'idrogeno atomico sono quelle a $21cm$.

L'osservazione della riga a $21cm$ \`e di fondamentale importanza nel capire presenza e concentrazioni di idrogeno atomico nel gas interstellare. Lo stato eccitato che ci interessa riguarda lo spin flip dell'elettrone passando da spin parallelo a $p$ ad antiparallelo a $p$. Il salto di energia \`e molto piccolo nell'ordine dei
\newl{
		\Delta E = 5.9 \times 10^{-6} eV.
}
La temperatura associata a questa energia \`e nell'ordine dei $0.07K$, quindi la radiazione di fondo cosmico a $2.7K$ \`e sufficiente a far si che la tansizione venga e sia possibile eccitare l'atomo.

La riga a $21cm$ \`e sufficientemente lunga da attraversare le polveri e raggiungere la terra dove \`e possibile rilevarla. Essendo cos\`i insensibile alle polveri permette di studiare la struttura e la rotazione deilla galassia. Andando a osservare lo split Zeeman sulla riga a $21cm$ \`e possibile anche avere informazioni sul campo magnetico galattico.

\subsection{Emissione di Idrogeno}





















